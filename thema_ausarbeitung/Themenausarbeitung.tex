\documentclass{article}
\usepackage[utf8]{inputenc}
\usepackage{hyperref}
\usepackage{graphicx}

\title{Automatisierung der Erkennung von Spitzen eines Mikromanipulators in einem Rasterelektronenmikroskop mithilfe von Neuronalen Netzen}
\author{David Kleindiek\\ Eberhard-Karls-Universität Tübingen
\and Betreuer: \\ Prof. Dr. Andreas Zell \\ Eberhard-Karls-Universität Tübingen
\and Dr. Matthias Kemmler\\ Kleindiek Nanotechnik GmbH}
\date{Januar 2022}

\begin{document}
	
	\maketitle
	
	\section{Einführung}
	\subsection{Einsatzfeld}
	Die Mikromanipulatoren von Kleindiek, werden meist zur Fehleranalyse bei Transistoren auf Computerchips eingesetzt.\\
	Dies geschieht hauptsächlich im Rasterelektronenmikroskop.
	\subsection{Geräte}
	Es wird ein \href{https://www.kleindiek.com/ps4.html}{PS4}/\href{https://www.kleindiek.com/ps8.html}{PS8} von Kleindiek in einem Gemini SEM von Zeiss betrieben.\\
	Die Anzahl der Nadeln kann variieren -zwischen 1 und 8.\\
	Dabei werden in stufenlos einstellbarer Vergrößerung vom mm bis in den $nm$ Bereich Bilder mit einer Auflösung von 1024x768 und ca. 3 FPS aufgenommen.\\
	Sie befinden sich im TIF Format.\\
	Im Header der Bilddateien befinden sich Informationen über den Zustand des Mikroskops (Parameter).
	\subsection{Probingablauf}
	Nach einmaligem Kalibrieren, können die Nadeln bis auf einen Bereich von $10\mu m$ zuverlässig zusammengefahren werden.\\
	Nun muss - unter ständiger visueller Kontrolle - langsam die Vergrößerung erhöht werden und die Nadeln händisch immer näher aneinander gefahren werden.\\
	Die XY Positionen der Nadeln liest man aus dem REM Bild ab. Die Höhe der Nadeln und Probe ermittelt man über den Fokus und die Tiefenschärfe des REMs.\\
	Hierbei ist darauf zu achten, dass sich die Nadeln nicht berühren, da sie sich sonnst verbiegen können.\\
	Sobald alle Nadeln auf gleicher Höhe und in einem gewünschten Radius sind, werden alle Nadeln etwa 10$\mu$um angehoben.\\
	Anschließend wird die Probe mittels der Stage des PSX auf etwa 10$\mu$um Abstand zu den Nadeln angehoben.\\
	Nun werden die Nadeln über die Kontaktflächen gebracht und vorsichtig auf der Probe abgesetzt.\\
	Den Touchdown erkennt man an einer Kontraständerung oder daran, dass sich die Nadelspitze um ein kleines Stück nach vorne bewegt.
	
	\section{Ziel der Arbeit}
	\subsection{Problemstellung}
A	Außerdem ist es eine große Verantwortung, da schnell ein großer Schaden entstehen kann.\\
	Viele Manöver könnten automatisiert werden und man könnte Unfälle verhindern, wenn man die Positionen der Nadelspitzen zuverlässig und automatisch aus den Bildern extrahieren könnte.\\
	\\
	Wenn dies gelingt, wäre ein weiterer Schritt, den Touchdown zuverlässig zu erkennen und zu detektieren, ob sich Spitzen im Kollisionskurs mit anderen Objekten befinden.
	\subsection{Idee}
	Trainieren eines Convolutional Neural Networks, um die Nadelkonturen und die Spitzenposition auszugeben.\\
	Hierzu könnte man ein Autoencoder Netzwerk darauf trainieren, ein Binärbild mit der Segmentierung von Spitzen und Probe zu erstellen.\\
	Ein weiteres Netzwerk könnte die Konturen zeichnen und die Spitze markieren.\\
	Die genaue Netzstruktur ist noch offen.\\
	Es kann auch ein "transfer learning" mit bekannten Netzen ausprobiert werden.

	\section{Erstellung eines Datensatzes}
	Es existieren bereits viele Videos und Bilder des Probingvorgangs.\\
	Diese können alle verwendet werden.
	\subsection{Labeling}
	Teil-automatisiertes labeling der Bilder mithilfe von Software.\\
	Beispielsweise:
	\begin{enumerate}
		\item \href{https://www.adobe.com/products/photoshop.html}{Adobe Photoshop } (automatische Objekt selektierung)
		\item \href{https://www.mathworks.com/help/vision/ref/imagelabeler-app.html}{Matlab - Image Labeler} (automatisches cross-frame labeling)
	\end{enumerate} 
	\subsection{Synthetische Daten}
	\subsubsection{Kombinierung von realen Daten}
	Verschiedene Bilder von Proben und Spitzen randomisiert übereinander gelegt um einen großen Datensatz zu generieren.\\
	Absprache mit Mitarbeiter welche Szenarien/Kombinationen realistisch sind.\\
	\subsubsection{Generierung von Daten}
	Szenerie des REM wird mithilfe von Software simuliert.\\
	So können unendlich viele Daten generiert werden.\\
	Beispielsweise:
	\begin{enumerate}
		\item \href{https://www.blender.org/}{Blender}
		\item \href{https://www.amagnm.com/pages/amag-simusem}{AMAG SimuSEM}
	\end{enumerate} 
	\subsection{Image augmentation}
	Bekannte Verfahren wie Transformationen der Bilder sowie Rauschfilter, und verändern von Helligkeit/Kontrast können verwendet werden, um die Anzahl der Daten zu erhöhen.
	
	\section{Anmerkungen}
	Es gibt viele weitere Informationen, die man neben den Bildern verwenden kann.\\
	Motorsteuerung (Richtung und grobe Distanz), vorherige Position, erwartete Spitzengröße (Vergrößerung).\\
	Ich werde eine Liste der verwendbaren Daten erstellen.\\
	\\
	Genaue Rahmenbedingungen sollten gemeinsam mit beiden Betreuern festgelegt werden.\\
	Bsp.: über welche Vergrößerung hinweg soll eine Spitze erkannt werden?
	
	
	

	
\end{document}
