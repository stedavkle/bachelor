\clearpage
\chapter{Verwandte Werke}

\section{Detektion und Segmentierung von Mitochondrien in Rasterelektronenmikroskop Bildern}
In der Arbeit \glqq Automatic Detection and Segmentation of Mitochondria from SEM Images using Deep Neural Network\grqq{} widmen sich Liu \textit{et al.} der Herausforderung, Mitochondrien in hochauflösenden REM-Bildern automatisch zu erkennen und zu segmentieren \cite{8512393}. Sie stellen fest, dass trotz erheblicher Fortschritte die Komplexität der Zellstrukturen, das Hintergrundrauschen und Bildartefakte diese Aufgabe nach wie vor erschweren.
Um dieses Problem zu lösen, schlagen die Autoren einen verbesserten Mask R-CNN Ansatz vor. Dieses Modell ist eine Weiterentwicklung der von He \textit{et al.} \cite{He_2017_ICCV} vorgestellten Mask R-CNN Architektur und beinhaltet neben der Erkennung und Segmentierung von Mitochondrien auch eine morphologische Verarbeitung und die Einbeziehung von Kontextinformationen zur Korrektur lokaler Fehleinschätzungen. Ein besonderes Merkmal dieses Ansatzes ist die Nutzung des Field of View (FoV) der Maskenverzweigung zur Erzeugung mehrerer Maskenausgaben in vier Richtungen, die für die abschließende Segmentierung kombiniert werden.

Die Validierungsergebnisse, basierend auf zwei gängigen Datensätzen, zeigen, dass der vorgeschlagene Ansatz eine vergleichbare Leistung wie die bisher besten Methoden erreicht. Darüber hinaus konnten Mitochondrien verschiedener Größen und Maßstäbe erfolgreich detektiert und segmentiert werden, wobei geeignete Nachbearbeitungsverfahren eine deutliche Verbesserung der Segmentierungsleistung ermöglichten.
Die vorgestellte Methode in der Arbeit unterstreicht die Effektivität von Deep Learning-Modellen, wie Mask R-CNN, bei der Bewältigung der Herausforderungen der Segmentierung in hochauflösenden REM-Bildern. Durch die gezielte Anpassung und Optimierung dieses Modells für die spezifischen Aufgaben der Mitochondrien-Erkennung konnten die Autoren bemerkenswerte Ergebnisse erzielen.
Diese Ergebnisse lassen darauf schließen, dass ähnliche Anpassungen des Modells es ebenso ermöglichen könnten, die Lokalisierung und Identifizierung von Messpitzen in REM-Bildern zu automatisieren und zu verbessern. Daher könnte die hier dargelegte Vorgehensweise einen vielversprechenden Ausgangspunkt für die Anpassung von Deep Learning-Modellen an die spezifischen Anforderungen der Messpitzen-Lokalisierung bieten.

\newpage
%\section{ZeroCostDL4Mic}
\section{FibeR-CNN}
Mit der Arbeit \glqq FibeR-CNN: Expanding Mask R-CNN to Improve Image-Based Fiber Analysis\grqq{} leisten Frei und Kruis einen wichtigen Beitrag zur Herausforderung der automatisierten Bildanalyse von faserförmigen Materialien in REM-Bildern. Sie identifizieren einen Mangel an effektiven automatisierten Algorithmen zur Bildannotation von überlappenden und verdeckten Fasern, die eine genaue Bestimmung der Faserlänge und -breite ermöglichen \cite{Frei_2021}. Bestehende Methoden wie CTFIRE sind nicht nur unzureichend, sondern auch sehr zeitaufwendig.
Um dieses Problem zu lösen, erweitern die Autoren die Mask R-CNN Architektur innerhalb des Detectron2 Frameworks und entwickeln einen spezifischen Ansatz für die Analyse von Faserbildern, den sie FibeR-CNN nennen \cite{wu2019detectron2}\cite{fiberrcnn}.

FibeR-CNN wurde so modifiziert, dass es zusätzlich zur Segmentierung auch die Breite und Länge der Fasern vorhersagen kann. Durch die Verwendung von CNNs erweist sich der Ansatz als robust gegenüber Änderungen der Bildbedingungen und benötigt nach dem Training keine Parameteranpassung durch den Benutzer. Somit ist es ein effektives Werkzeug für die automatische, bildbasierte Faserformanalyse.
Obwohl FibeR-CNN nicht direkt als Grundlage für diese Arbeit dient, bietet es wertvolle Einblicke, wie Mask R-CNN für spezifische Anforderungen in der REM-Bildsegmentierung modifiziert werden kann. Darüber hinaus bestätigt es, dass Mask R-CNN effektiv zur Segmentierung komplexer Strukturen in REM-Bildern eingesetzt werden kann, was Parallelen zur Aufgabe der Spitzenerkennung aufweist.
\section{Detectron2}
Detectron2 ist ein Open-Source-Software-Framework, das von der Facebook AI Research Group entwickelt wurde. Es bietet eine flexible, modulare und erweiterbare Architektur, die eine einfache Anpassung und Erweiterung verschiedener Modelle ermöglicht und die Entwicklung und das Experimentieren mit neuen Modellen und Algorithmen erleichtert. Detectron2 enthält Implementierungen der neuesten Objekterkennungs- und Segmentierungsalgorithmen, einschließlich Mask R-CNN und Faster R-CNN, und hat den Vorteil, dass es eine Reihe von vortrainierten Modellen bietet, die auf großen und unterschiedlichen Datensätzen wie COCO trainiert wurden. Durch die Verwendung von Detectron2 kann das Modell von diesen vortrainierten Gewichten profitieren, was den Trainingsprozess beschleunigt und die Modellleistung verbessert. Die in Detectron2 integrierten Werkzeuge für Training, Inferenz und Evaluierung sowie die umfangreichen Datenverarbeitungsfunktionen machen Detectron2 zu einem umfassenden Werkzeug, das den gesamten Prozess der Modellerstellung und -evaluierung erleichtert. Detectron2 bietet eine nahtlose Integration und Interoperabilität mit dem PyTorch Ökosystem \cite{wu2019detectron2}.