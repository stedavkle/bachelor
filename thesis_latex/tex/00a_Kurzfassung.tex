Die Halbleiterindustrie, eine sich ständig weiterentwickelnde und komplexe Branche, die sich auf Strukturen unter 10 Nanometern reduziert hat, steht vor ständigen Herausforderungen bei der Fehleranalyse der entwickelten Strukturen. Insbesondere die zuverlässige Lokalisierung von Messspitzen im Nanoprobing ist entscheidend, um die Effizienz durch Automatisierung zu steigern und gleichzeitig Beschädigungen zu vermeiden. Diese Bachelorarbeit beschäftigt sich mit der Entwicklung und Implementierung von Deep Learning-Verfahren, insbesondere des Mask R-CNN Modells, zur Detektion von Messspitzen in rasterelektronenmikroskopischen Bildern.

Durch die Entwicklung und Implementierung von Schnittstellen zu Kleindiek Manipulatoren und ZEISS Mikroskopen in Python wird die Datenerfassung automatisiert und ein umfangreicher Datensatz für Nanoprobing erzeugt. Das Modell wird angepasst und trainiert, um die spezifischen Herausforderungen des Bildrauschens und der Formenvielfalt der Messspitzen zu überwinden.

Die Arbeit gibt einen detaillierten Einblick in den Nanoprobing-Prozess, und beinhaltet die umfassende Implementierung, das Training und die Evaluierung des Mask R-CNN Modells. Die Ergebnisse dieser Arbeit deuten darauf hin, dass Deep Learning eine vielversprechende Lösung für die Prozessautomatisierung in der Fehleranalyse in der Halbleiterindustrie ist und bieten eine solide Grundlage für die weitere Automatisierung und Verbesserung der Fehleranalyse, die Modelle zu optimieren und die Technologie in bestehende Prozesse zu integrieren.